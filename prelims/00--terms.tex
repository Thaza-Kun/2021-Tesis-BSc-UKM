\chapter{Senarai Istilah}

\begin{center}
\doublespacing
\begin{tabular}{l@{\hspace{3em}}p{.6\textwidth}}
%Elektrolemah & Electroweak\\
%Empat Daya Asasi Alam & The Four Fundamental Forces of Nature\\
%Fungsian & Functional\\
Garis dunia & Garis yang dihasilkan oleh zarah merentasi ruang dan masa. (\textit{World line})\\
%Graviti Kuantum Lingkaran & Loop Quantum Gravity (LQG)\\
%Indeks bebas & Indeks yang tiada pasangan. (\textit{Free index})\\
%Indeks semu & Indeks yang mempunyai pasangan. (\textit{Dummy index})\\
Kalis tukar koordinat & Sifat persamaan yang tidak bergantung kepada koordinat yang dipilih. (\textit{Coordinate invariant})\\
Kalis tukar parameter & Sifat persamaan yang tidak bergantung kepada parameter yang dipilih. (\textit{Reparametrization Invariant})\\
%Kalkulus Ubahan & Calculus of Variation\\
%Kamiran & Integration\\
%Kamiran Cebis Demi Cebis & Integration by Parts\\
%Kekalisan & Invariance\\
%Kekalisan Koordinat & Coordinate Invariance\\
%Kekalisan Parameter & Reparametrization Invariance\\
%Kuark Aneh & Strange Quark, $s$\\
%Kuark Atas & Top Quark, $t$\\
%Kuark Bawah & Bottom Quark, $b$\\
%Kuark Cun & Charm Quark, $c$\\
%Kuark Naik & Up Quark, $u$\\
%Kuark Turun & Down Quark, $d$\\
Lembar dunia & Lembaran yang dihasilkan oleh tetangsi yang bergerak merentasi ruang dan masa. (\textit{World sheet})\\
%Mahakecil & Infinitesimal\\
%Matra & Nilai yang menyatakan darjah kebebasan objek dalam ruang--masa. (\textit{Dimension})\\
Model Piawai & Model yang memuatkan zarah-zarah fermion dan boson. (\textit{Standard Model})\\
%Pekalis Lorentz & Nilai yang tidak berubah jika dikenakan penjelmaan Lorentz. (\textit{Lorentz invariant})\\
%Penjumlahan Tersirat & Implied Summation\\
%Petua rantai & Chain rule\\
Prinsip tindakan pegun & Prinsip yang mengatakan bahawa laluan zarah akan sentiasa meminimakan tindakannya. (\textit{Principle of stationary action})\\
%Tatatanda Einstein & Tatatanda dalam kalkulus tensor yang mewujudkan penjumlahan sekiranya wujud indeks semu. (\textit{Einstein Notation})\\
Tindakan & Nilai yang menentukan laluan suatu jasad. (\textit{Action}) \\
%Teori tetangsi & Teori yang menganggap bahawa semua benda diperbuat daripada tetangsi bermatra--1. (\textit{String theory}) \\
%Teori segala jasad & Suatu usaha untuk menerangkan semua benda dalam satu teori yang menyeluruh. (\textit{Theory of everything})\\
%Unit Alami & Natural Units\\
%Vektor Asasi & Basis Vector\\
\end{tabular}
\end{center}